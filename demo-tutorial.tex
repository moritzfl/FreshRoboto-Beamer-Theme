%!TEX program = xelatex

% Name           : demo.tex
% Author         : Moritz Flöter
% Version        : 1.0
% Created on     : 17.04.2016
% Last Edited on : 03.04.2017
% Copyright      : Copyright (c) 2016-2017 by Moritz Flöter.
% Based on       : HSRM-Theme from Benjamin Weiss
% License        : This file may be distributed and/or modified under the
%                  GNU Public License.
% Description    : HSRM beamer theme demonstration. Also includes a short 
%                  Tutorial regarding the beamer class.

%--------------------------------------------------------------------------
% aspectratio -> 4:3 -> 43, 16:9 -> 169, 16:10 -> 1610 etc.
%--------------------------------------------------------------------------
\documentclass[compress, aspectratio=1610, noserifmath]{beamer}
%--------------------------------------------------------------------------
% Common packages
%--------------------------------------------------------------------------
\usepackage[german]{babel}

\usepackage{graphicx}
\usepackage{multicol}
% Erweiterte Tabellenfunktionen
\usepackage{tabularx,ragged2e}
\usepackage{booktabs}
% Listingserweiterung
\usepackage{listings}
\lstset{ %
language=[LaTeX]TeX,
basicstyle=\normalsize\ttfamily,
keywordstyle=,
numbers=left,
numberstyle=\tiny\ttfamily,
stepnumber=1,
showspaces=false,
showstringspaces=false,
showtabs=false,
breaklines=true,
frame=tb,
framerule=0.5pt,
tabsize=4,
framexleftmargin=0.5em,
framexrightmargin=0.5em,
xleftmargin=0.5em,
xrightmargin=0.5em
}

%--------------------------------------------------------------------------
% Load theme
%--------------------------------------------------------------------------
\usetheme{freshroboto}

% must be loaded after theme
\usepackage{tikz}
\usetikzlibrary{mindmap,backgrounds}

%--------------------------------------------------------------------------
% General presentation settings
%--------------------------------------------------------------------------
\title{Beamer Theme}
\subtitle{Demonstration und kurze Einführung in Beamer}
\date{Letztes Update: \today}
\author{Max Mustermann}
\institute{Institut für WasAuchImmer \\ \textbf{Universität Studienort}}

%--------------------------------------------------------------------------
% Notes settings
%--------------------------------------------------------------------------
\setbeameroption{show notes}

\begin{document}
%--------------------------------------------------------------------------
% Titlepage
%--------------------------------------------------------------------------

\maketitle

%\begin{frame}[plain]
%	\titlepage
%\end{frame}

%--------------------------------------------------------------------------
% Table of contents
%--------------------------------------------------------------------------
\section*{Gliederung}
\begin{frame}{Gliederung}
	% hideallsubsections ist empfehlenswert für längere Präsentationen
	\tableofcontents[hideallsubsections]

\end{frame}

%--------------------------------------------------------------------------
% Content
%--------------------------------------------------------------------------
\section{Einleitung}

\begin{frame}{Was ist Beamer?}
	Die Beamer Klassen für \LaTeX\ dienen zur Erstellung von Präsentationen, welche mit einem Beamer vorgeführt werden sollen. Das Textsatzsystem erzeugt dazu PDF Dateien, die von einer großen Anzahl an Programmen gezeigt werden können.
	
	Das hier vorgestellte Theme für Beamer macht das Erstellen von Folien(Grundkenntnisse in \LaTeX\ vorausgesetzt) zu einem Kinderspiel.
\end{frame}

\begin{frame}{Systemvoraussetzungen}
	Um erfolgreich Präsentationen mit diesem Theme erstellen zu können, sind folgende Voraussetzungen vom System zu erfüllen:
	\begin{itemize}
		\item Zum Setzen der Folien muss LuaLaTeX oder XeLaTeX verwendet werden.
		\item Neben einigen Standardpaketen müssen die Pakete \texttt{beamer}, \texttt{pgf} und \texttt{xcolor} installiert sein.
		\item Die Vorlage verwendet die Roboto-Schriftarten von Google, welche unter der Apache License Version 2.0 stehen und ohne Installation aus dem Ordner template/font eingebunden werden..
	\end{itemize}
\end{frame}

\section{Tutorial}
\begin{frame}[containsverbatim]{Grundaufbau des Dokuments}
Der Grundaufbau ist einfach:
\begin{lstlisting}
\documentclass[compress]{beamer}
% Theme laden
\usetheme{hsrm}
% Allgemeine Präsentationseinstellungen 
\title{Titel der Präsentation}
\subtitle{Untertitel der Präsentation}
\author{Ihr Name}
\institute{Studienbereich\\Hochschule {\Medium RheinMain}}
\begin{document}
% Folien
\end{document}
\end{lstlisting}
\end{frame}

\begin{frame}{Themeoptionen}
Um die Darstellung der Präsentation anzupassen können die folgenden Optionen gewählt werden.
\begin{table}[]
	\begin{tabularx}{\linewidth}{l>{\raggedright}X}
		\toprule
		\textbf{Option}			& \textbf{Auswirkung} \tabularnewline
		\midrule
		\texttt{noserifmath}		& Formeln werden ebenfalls serifenlos gesetzt. \tabularnewline
		\texttt{nosectionpages} & Die Sektionseinleitungsseiten werden ausgeblendet.\tabularnewline
		\bottomrule
	\end{tabularx}
	\label{tab:options}
\end{table}
\end{frame}

\begin{frame}{Primärfarben}									
Die Primärfarben des Designs sind im Template hinterlegt.


\begin{multicols}{2}

\setbeamercolor{hsrmRedDemo}{fg=hsrmRed,bg=white}
\begin{beamercolorbox}[wd=\linewidth,ht=2ex,dp=0.7ex]{hsrmRedDemo}
	\texttt{hsrmRed}
\end{beamercolorbox}
\setbeamercolor{hsrmRedDarkDemo}{fg=hsrmRedDark,bg=white}
\begin{beamercolorbox}[wd=\linewidth,ht=2ex,dp=0.7ex]{hsrmRedDarkDemo}
	\texttt{hsrmRedDark}
\end{beamercolorbox}
\setbeamercolor{hsrmWarmGreyDarkDemo}{fg=hsrmWarmGreyDark,bg=white}
\begin{beamercolorbox}[wd=\linewidth,ht=2ex,dp=0.7ex]{hsrmWarmGreyDarkDemo}
	\texttt{hsrmWarmGreyDark}
\end{beamercolorbox}
\setbeamercolor{hsrmWarmGreyLightDemo}{fg=hsrmWarmGreyLight,bg=white}
\begin{beamercolorbox}[wd=\linewidth,ht=2ex,dp=0.7ex]{hsrmWarmGreyLightDemo}
	\texttt{hsrmWarmGreyLight}
\end{beamercolorbox}

\setbeamercolor{hsrmRedDemoBg}{fg=white,bg=hsrmRed}
\begin{beamercolorbox}[wd=\linewidth,ht=2ex,leftskip=.5ex,dp=0.7ex]{hsrmRedDemoBg}
	\texttt{hsrmRed}
\end{beamercolorbox}
\setbeamercolor{hsrmRedDarkDemoBg}{fg=white,bg=hsrmRedDark}
\begin{beamercolorbox}[wd=\linewidth,ht=2ex,leftskip=.5ex,dp=0.7ex]{hsrmRedDarkDemoBg}
	\texttt{hsrmRedDark}
\end{beamercolorbox}
\setbeamercolor{hsrmWarmGreyDarkDemo}{fg=white,bg=hsrmWarmGreyDark}
\begin{beamercolorbox}[wd=\linewidth,ht=2ex,leftskip=.5ex,dp=0.7ex]{hsrmWarmGreyDarkDemo}
	\texttt{hsrmWarmGreyDark}
\end{beamercolorbox}
\setbeamercolor{hsrmWarmGreyLightDemo}{fg=white,bg=hsrmWarmGreyLight}
\begin{beamercolorbox}[wd=\linewidth,ht=2ex,leftskip=.5ex,dp=0.7ex]{hsrmWarmGreyLightDemo}
	\texttt{hsrmWarmGreyLight}
\end{beamercolorbox}

\end{multicols}

\end{frame}

\begin{frame}{Sekundärfarben}
\begin{multicols}{2}

\setbeamercolor{hsrmSec1Demo}{fg=hsrmSec1,bg=white}
\begin{beamercolorbox}[wd=\linewidth,ht=2ex,dp=0.7ex]{hsrmSec1Demo}
	\texttt{hsrmSec1}
\end{beamercolorbox}
\setbeamercolor{hsrmSec1DarkDemo}{fg=hsrmSec1Dark,bg=white}
\begin{beamercolorbox}[wd=\linewidth,ht=2ex,dp=0.7ex]{hsrmSec1DarkDemo}
	\texttt{hsrmSec1Dark}
\end{beamercolorbox}
\setbeamercolor{hsrmSec1CompDemo}{fg=hsrmSec1Comp,bg=white}
\begin{beamercolorbox}[wd=\linewidth,ht=2ex,dp=0.7ex]{hsrmSec1CompDemo}
	\texttt{hsrmSec1Comp}
\end{beamercolorbox}
\setbeamercolor{hsrmSec1CompDarkDemo}{fg=hsrmSec1CompDark,bg=white}
\begin{beamercolorbox}[wd=\linewidth,ht=2ex,dp=0.7ex]{hsrmSec1CompDarkDemo}
	\texttt{hsrmSec1CompDark}
\end{beamercolorbox}

\setbeamercolor{hsrmSec2Demo}{fg=hsrmSec2,bg=white}
\begin{beamercolorbox}[wd=\linewidth,ht=2ex,dp=0.7ex]{hsrmSec2Demo}
	\texttt{hsrmSec2}
\end{beamercolorbox}
\setbeamercolor{hsrmSec2DarkDemo}{fg=hsrmSec2Dark,bg=white}
\begin{beamercolorbox}[wd=\linewidth,ht=2ex,dp=0.7ex]{hsrmSec2DarkDemo}
	\texttt{hsrmSec2Dark}
\end{beamercolorbox}
\setbeamercolor{hsrmSec2CompDemo}{fg=hsrmSec2Comp,bg=white}
\begin{beamercolorbox}[wd=\linewidth,ht=2ex,dp=0.7ex]{hsrmSec2CompDemo}
	\texttt{hsrmSec2Comp}
\end{beamercolorbox}
\setbeamercolor{hsrmSec2CompDarkDemo}{fg=hsrmSec2CompDark,bg=white}
\begin{beamercolorbox}[wd=\linewidth,ht=2ex,dp=0.7ex]{hsrmSec2CompDarkDemo}
	\texttt{hsrmSec2CompDark}
\end{beamercolorbox}

\setbeamercolor{hsrmSec3Demo}{fg=hsrmSec3,bg=white}
\begin{beamercolorbox}[wd=\linewidth,ht=2ex,dp=0.7ex]{hsrmSec3Demo}
	\texttt{hsrmSec3}
\end{beamercolorbox}
\setbeamercolor{hsrmSec3DarkDemo}{fg=hsrmSec3Dark,bg=white}
\begin{beamercolorbox}[wd=\linewidth,ht=2ex,dp=0.7ex]{hsrmSec3DarkDemo}
	\texttt{hsrmSec3Dark}
\end{beamercolorbox}
\setbeamercolor{hsrmSec3CompDemo}{fg=hsrmSec3Comp,bg=white}
\begin{beamercolorbox}[wd=\linewidth,ht=2ex,dp=0.7ex]{hsrmSec3CompDemo}
	\texttt{hsrmSec3Comp}
\end{beamercolorbox}
\setbeamercolor{hsrmSec3CompDarkDemo}{fg=hsrmSec3CompDark,bg=white}
\begin{beamercolorbox}[wd=\linewidth,ht=2ex,dp=0.7ex]{hsrmSec3CompDarkDemo}
	\texttt{hsrmSec3CompDark}
\end{beamercolorbox}

\setbeamercolor{hsrmSec1DemoBg}{fg=white,bg=hsrmSec1}
\begin{beamercolorbox}[wd=\linewidth,ht=2ex,leftskip=.5ex,dp=0.7ex]{hsrmSec1DemoBg}
	\texttt{hsrmSec1}
\end{beamercolorbox}
\setbeamercolor{hsrmSec1DarkDemoBg}{fg=white,bg=hsrmSec1Dark}
\begin{beamercolorbox}[wd=\linewidth,ht=2ex,leftskip=.5ex,dp=0.7ex]{hsrmSec1DarkDemoBg}
	\texttt{hsrmSec1Dark}
\end{beamercolorbox}
\setbeamercolor{hsrmSec1CompDemoBg}{fg=white,bg=hsrmSec1Comp}
\begin{beamercolorbox}[wd=\linewidth,ht=2ex,leftskip=.5ex,dp=0.7ex]{hsrmSec1CompDemoBg}
	\texttt{hsrmSec1Comp}
\end{beamercolorbox}
\setbeamercolor{hsrmSec1CompDarkDemoBg}{fg=white,bg=hsrmSec1CompDark}
\begin{beamercolorbox}[wd=\linewidth,ht=2ex,leftskip=.5ex,dp=0.7ex]{hsrmSec1CompDarkDemoBg}
	\texttt{hsrmSec1CompDark}
\end{beamercolorbox}

\setbeamercolor{hsrmSec2DemoBg}{fg=white,bg=hsrmSec2}
\begin{beamercolorbox}[wd=\linewidth,ht=2ex,leftskip=.5ex,dp=0.7ex]{hsrmSec2DemoBg}
	\texttt{hsrmSec2}
\end{beamercolorbox}
\setbeamercolor{hsrmSec2DarkDemoBg}{fg=white,bg=hsrmSec2Dark}
\begin{beamercolorbox}[wd=\linewidth,ht=2ex,leftskip=.5ex,dp=0.7ex]{hsrmSec2DarkDemoBg}
	\texttt{hsrmSec2Dark}
\end{beamercolorbox}
\setbeamercolor{hsrmSec2CompDemoBg}{fg=white,bg=hsrmSec2Comp}
\begin{beamercolorbox}[wd=\linewidth,ht=2ex,leftskip=.5ex,dp=0.7ex]{hsrmSec2CompDemoBg}
	\texttt{hsrmSec2Comp}
\end{beamercolorbox}
\setbeamercolor{hsrmSec2CompDarkDemoBg}{fg=white,bg=hsrmSec2CompDark}
\begin{beamercolorbox}[wd=\linewidth,ht=2ex,leftskip=.5ex,dp=0.7ex]{hsrmSec2CompDarkDemoBg}
	\texttt{hsrmSec2CompDark}
\end{beamercolorbox}

\setbeamercolor{hsrmSec3Demo}{fg=white,bg=hsrmSec3}
\begin{beamercolorbox}[wd=\linewidth,ht=2ex,leftskip=.5ex,dp=0.7ex]{hsrmSec3Demo}
	\texttt{hsrmSec3}
\end{beamercolorbox}
\setbeamercolor{hsrmSec3DarkDemo}{fg=white,bg=hsrmSec3Dark}
\begin{beamercolorbox}[wd=\linewidth,ht=2ex,leftskip=.5ex,dp=0.7ex]{hsrmSec3DarkDemo}
	\texttt{hsrmSec3Dark}
\end{beamercolorbox}
\setbeamercolor{hsrmSec3CompDemo}{fg=white,bg=hsrmSec3Comp}
\begin{beamercolorbox}[wd=\linewidth,ht=2ex,leftskip=.5ex,dp=0.7ex]{hsrmSec3CompDemo}
	\texttt{hsrmSec3Comp}
\end{beamercolorbox}
\setbeamercolor{hsrmSec3CompDarkDemo}{fg=white,bg=hsrmSec3CompDark}
\begin{beamercolorbox}[wd=\linewidth,ht=2ex,leftskip=.5ex,dp=0.7ex]{hsrmSec3CompDarkDemo}
	\texttt{hsrmSec3CompDark}
\end{beamercolorbox}

\end{multicols}
\end{frame}

\begin{frame}[containsverbatim]{Folienstruktur}
Strukturiert wird in Beamer wie in \LaTeX\ üblich mittels \lstinline!section!, \lstinline!subsection!, usw. Für Folien ist die \lstinline!frame! Umgebung definiert.

Der Folientitel kann direkt an die \lstinline!frame! Umgebung übergeben werden oder mittels \lstinline!\frametitle{Folientitel}! innerhalb der Umgebung gesetzt werden.
\begin{lstlisting}
\section{Meine Sektion}
\subsection{Meine Subsektion}
\begin{frame}
\frametitle{Folientitel}
% Folieninhalt
\end{frame}
\end{lstlisting}
\end{frame}

\begin{frame}[containsverbatim]{Titelseite und Inhaltsverzeichnis}
Die Titelseite erzeugt man mit 
\begin{lstlisting}
\maketitle
\end{lstlisting}
Und das Inhaltsverzeichnis mit
\begin{lstlisting}
\begin{frame}{Gliederung}
	\tableofcontents[hideallsubsections]
\end{frame}
\end{lstlisting}
Die Option \lstinline!hideallsubsections! bietet sich bei längeren Präsentationen an, um das Inhaltsverzeichnis kompakt zu halten.
\end{frame}

\subsection{Aufzählungen}
\begin{frame}[containsverbatim]{Aufzählungen}
Aufzählungen sind mit der \lstinline!enumerate! und der \lstinline!itemize! Umgebung möglich.
\begin{enumerate}
	\item Punkt 1
	\item Punkt 2
	\begin{itemize}
		\item Punkt 1
		\item Punkt 2
	\end{itemize}
	\item Punkt 3
\end{enumerate}
\end{frame}

\subsection{Hervorhebungen}
\begin{frame}[containsverbatim]{Hervorhebungen}
In der Beamer Klasse ist die Funktion \lstinline!\alert! definiert, um einzelne Wörter hervorzuheben. Beispiel:
\begin{itemize}
	\item \alert{hervorgehobener Text}
\end{itemize}

Darüber hinaus sind im Theme noch folgende Zusatzfunktionen für den Schnellzugriff definiert :
\begin{itemize}
	\item \lstinline!\quoted{Text}! \hspace{5em} \quoted{Text}
	\item \lstinline!\doublequoted{Text}! \hspace{1em} \doublequoted{Text}
	\item \lstinline!\cmark!   \hspace{5em} \cmark 
	\item \lstinline!\xmark!   \hspace{5em} \xmark 	
\end{itemize}
\end{frame}

\subsection{Blockstrukturen}
\begin{frame}[containsverbatim]{Einfacher Block mit Aufzählung}
Zur Strukturierung sind in Beamer Blockumgebungen definiert.
\begin{block}{Block mit einer Aufzählung}
	\begin{itemize}
		\item Punkt 1
		\item Punkt 2
	\end{itemize}
\end{block}
\begin{lstlisting}
\begin{block}{Block mit einer Aufzählung}
	\begin{itemize}
		\item Punkt 1
		\item Punkt 2
	\end{itemize}
\end{block}
\end{lstlisting}
\end{frame}

\begin{frame}[containsverbatim]{Alert Block}
\begin{alertblock}{Alert Block}
	Ein Alert Block wird mit der ersten Primärfarbe eingefärbt.
\end{alertblock}
\begin{lstlisting}
\begin{alertblock}{Alert Block}
Ein Alert Block wird mit der ersten Primärfarbe eingefärbt.
\end{alertblock}
\end{lstlisting}
\end{frame}

\begin{frame}[containsverbatim]{Example Block}
\begin{exampleblock}{Example Block}
	Ein Example Block wird mit der ersten Sekundärfarbe eingefärbt.
\end{exampleblock}
\begin{lstlisting}
\begin{exampleblock}{Example Block}
Ein Example Block wird mit der ersten Sekundärfarbe eingefärbt.
\end{exampleblock}
\end{lstlisting}
\end{frame}

\begin{frame}[containsverbatim]{Block mit anderer Farbe}
\begingroup
\setbeamercolor{block title}{bg=hsrmSec2Dark}
\setbeamercolor{block body}{bg=hsrmSec2}
\begin{block}{Block mit anderer Farbe}
	In diesem Block wird eine weitere Sekundärfarbe verwendet.
\end{block}
\endgroup
\begin{lstlisting}
\begingroup
\setbeamercolor{block title}{bg=hsrmSec2Dark}
\setbeamercolor{block body}{bg=hsrmSec2}
\begin{block}{Block mit anderer Farbe}
	In diesem Block wird ...
\end{block}
\endgroup
\end{lstlisting}
\end{frame}

\section{Beispielfolien}
\begin{frame}{Weitere Beispiele}
Nachfolgend sind weitere Beispielfolien ohne zusätzliche Erläuterung angehängt.

Schauen Sie einfach in den Quelltext, um zu sehen wie die Folien erstellt wurden.
\end{frame}
\subsection{Abbildungen}
\begin{frame}{Foto mit Copyright}
	\begin{figure}
		\centering
		\includegraphicscopyright[width=60mm, height=60mm]{template/img/logo.eps}{Copyright by me, License}
	\end{figure}
\end{frame}


\subsection{Tabellen}
\begin{frame}{Tabelle}
\begin{table}[]
	\caption{Selection of window function and their properties}
	\begin{tabular}[]{lrrr}
		\toprule
		\textbf{Window}			& \multicolumn{1}{c}{\textbf{First side lobe}}	
		                    & \multicolumn{1}{c}{\textbf{3\,dB bandwidth}}
		                    & \multicolumn{1}{c}{\textbf{Roll-off}} \\
		\midrule
		Rectangular				& 13.2\,dB	& 0.886\,Hz/bin	& 6\,dB/oct		\\[0.25em]
		Triangular				& 26.4\,dB	& 1.276\,Hz/bin	& 12\,dB/oct	\\[0.25em]
		Hann					& 31.0\,dB	& 1.442\,Hz/bin	& 18\,dB/oct	\\[0.25em]
		Hamming					& 41.0\,dB	& 1.300\,Hz/bin	& 6\,dB/oct		\\
		\bottomrule
	\end{tabular}
	\label{tab:WindowFunctions}
\end{table}
\end{frame}

\subsection{Formeln}
\begin{frame}{Formeln}
\begin{block}{Fourierintegral}
\begin{equation*}
F(\textrm{j}\omega) = \int\limits_{-\infty}^{\infty} f(t)\cdot\textrm{e}^{-\textrm{j}\omega t} dt
\end{equation*}
\end{block}
\begin{block}{Fakultät}
\begin{equation*}
	n! = 1\cdot 2 \cdot 3 \cdot\ldots\cdot n = \prod_{k=1}^n k
\end{equation*}
\end{block}
\end{frame}

\subsection{Fußnoten}
\begin{frame}{Fußnoten}
	Lorem ipsum dolor sit amet, consetetur sadipscing elitr, sed diam nonumy eirmod tempor invidunt ut labore et dolore magna aliquyam erat, sed diam voluptua. At vero eos et accusam et justo duo dolores et ea rebum. Stet clita kasd gubergren, no sea takimata sanctus est Lorem ipsum dolor sit amet. Lorem \footnote{Lorem ipsum dolor sit amet} ipsum dolor sit amet, consetetur sadipscing elitr, sed diam nonumy eirmod tempor invidunt ut labore et dolore magna aliquyam erat, sed diam voluptua. At vero eos et accusam et justo duo dolores et ea rebum. Stet clita kasd gubergren, no sea takimata sanctus est Lorem ipsum dolor sit amet.
\end{frame}

\subsection{Notizen}
\begin{frame}{Folie mit dazugehöriger Notizfolie}
    Für das Publikum ist diese Folie.

    Für ihre Präsentation bieten sich folgende Programme an:
    \begin{itemize}
        \item OSX
        \begin{itemize}
            \item pdf to keynote \\ \url{https://www.cs.hmc.edu/~oneill/freesoftware/pdftokeynote.html}    
            \item Splitshow \\ \url{https://github.com/mpflanzer/splitshow}
        \end{itemize}
        \item Windows
        \begin{itemize}
            \item Adobe Reader \\ \url{http://www.wikihow.com/View-a-PDF-Document-in-Full-Screen-View} 
        \end{itemize}
    \end{itemize}

\end{frame}

\note{
    Für Ihre Notizen zum Vortrag vewenden Sie diese Folie.
    
	Für ihre Präsentation bieten sich folgende Programme an:
	  \begin{itemize}
        \item OSX
        \begin{itemize}
            \item pdf to keynote \\ \url{https://www.cs.hmc.edu/~oneill/freesoftware/pdftokeynote.html}    
            \item Splitshow \\ \url{https://github.com/mpflanzer/splitshow}
        \end{itemize}
        \item Windows
        \begin{itemize}
            \item Adobe Reader \\ \url{http://www.wikihow.com/View-a-PDF-Document-in-Full-Screen-View} 
        \end{itemize}
    \end{itemize}
}

\subsection{Spalten}
\begin{frame}{Zwei Spalten}
	\begin{multicols}{2}
		Lorem ipsum dolor sit amet, consetetur sadipscing elitr, sed diam nonumy eirmod tempor invidunt ut labore et dolore magna aliquyam erat, sed diam voluptua. At vero eos et accusam et justo duo dolores et ea rebum. Stet clita kasd gubergren, no sea takimata sanctus est Lorem ipsum dolor sit amet.
		\begin{itemize}
        	\item ein Eintrag
        	\item noch ein Eintrag
		\end{itemize}
	\end{multicols}
\end{frame}

\begin{frame}{Spaltenumbruch}
	\begin{multicols}{2}
		Lorem ipsum dolor sit amet, consetetur sadipscing elitr, sed diam nonumy eirmod tempor invidunt ut labore et dolore magna aliquyam erat, sed diam voluptua. At vero eos et accusam et justo duo dolores et ea rebum. Stet clita kasd gubergren, no sea takimata sanctus est Lorem ipsum dolor sit amet.
		\columnbreak
		\begin{itemize}
        	\item ein Eintrag
        	\item noch ein Eintrag
		\end{itemize}
	\end{multicols}
\end{frame}

\begin{frame}{Literaturverzeichnis}
	\begin{thebibliography}{10}
    
	\beamertemplatebookbibitems
	\bibitem{Oppenheim2009}
	Alan~V.~Oppenheim
	\newblock \doublequoted{Discrete-Time Signal Processing}
	\newblock Prentice Hall Press, 2009

	\beamertemplatearticlebibitems
	\bibitem{EBU2011}
	European~Broadcasting~Union
	\newblock \doublequoted{Specification of the Broadcast Wave Format (BWF)}
	\newblock 2011
  \end{thebibliography}
\end{frame}

\section{Ausblick}
\begin{frame}{Bekannte Fehler}
	\begin{itemize}
		\item Theme ist momentan noch in einer einzelnen sty-Datei. Diese sollte unterteilt werden in einzelne Dateien für Schrift, Farbe usw.
	\end{itemize}
\end{frame}

\begin{frame}{Fragen, Anmerkungen, Kontakt}
	Das Theme steht unter der \quoted{GNU Public License}. Es darf also weitergegeben und modifiziert werden, sofern die Lizenzart beibehalten wird.
	
	Für Fragen und Anmerkungen stehe ich gerne zur Verfügung.
	\begin{itemize}
		\item $\rightarrow$ github verwenden
	\end{itemize}
\end{frame}

\end{document}






